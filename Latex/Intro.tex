\documentclass[11pt,a4paper]{article}
\usepackage[utf8]{inputenc}
\usepackage[french]{babel}
\usepackage{graphicx}
\usepackage{color}
\usepackage{setspace}
\usepackage{enumitem}
\usepackage[left=2.8cm,right=2.8cm,top=2.5cm,bottom=2.5cm]{geometry}

\title{Introduction}
\author{Guilmot Gautier \\ Hansart Charlotte \\ Lardinois Maxence \\ Mulders Zélie \\ Tihon Simon  }
\date{Avril 2015}

\begin{document}
\maketitle

Au cours de ce deuxième rapport, c'est la modélisation des données qui était au programme. Nous avons vu que, avant d'arriver à la remise d'un programme Android, le développement de celui-ci respecte un processus comme suit : tout commence par l'analyse du problème et des besoin, sans rechercher immédiatement une solution. C'est en constituant des user stories (les histoires utilisateur) qu'on a pu accomplir cette étape. S'en suivit l'étape de conception, celle où nous avons dû nous intéresser à comment nous allions procéder et comment nous pourrions contruire le système. C'est cette étape-là que nous avons regardé en détail au cours de ce rapport. Avant de penser à la construction du système via les diagrammes de classes UML et les diagrammes de séquences UML, nous avons d'abord constitué des cartes CRC qui nous ont servi d'intermédiaire de l'analyse à la conception.
Les deux dernières étapes, l'implémentation et les tests (et documentation) du programme, seront alors bien préparées et nous auront moins de mal à nous y attaquer.


\end{document}