\documentclass{article}
\usepackage[utf8]{inputenc}
\usepackage[french]{babel}
\usepackage{graphicx}
\usepackage[left=2.8cm,right=2.8cm,top=2.5cm,bottom=2.5cm]{geometry}


\title{Extensions : choix et description}

\author{Guilmot Gautier \\ Hansart Charlotte \\ Lardinois Maxence \\ Mulders Zélie \\ Tihon Simon  }

\date{February 2015}

\begin{document}

\maketitle

\section{Extension 1 : rapports}
La première extension que nous implémenterons permettra de visualiser différents rapports et statistiques, portant sur différents aspects du fonctionnement du bar.

Exemples de rapports : 
\begin{itemize}
\item le client le plus buveur.
\item chiffre d'affaire pour une période donnée (jour/semaine/mois) : on déduit le coût (achat de boissons, salaire des serveurs, loyer, ...) de la recette.
\item courbe de vente de tel ou tel produit, de telle ou telle catégorie (soft, bière, vin, ...).
\item comparaison du taux de vente de tel ou tel produit.
\item "performances" des serveurs : comparer le nombre de commandes servies par serveur.
\item comparer la rentabilité des différentes périodes de la journée et/ou de la semaine : y a-t-il plus de monde l'après-midi ou le soir ? Le weekend ou le vendredi ? 
\end{itemize} 
\section{Extension 2 : options gestionnaire}
Cette seconde extension permettra au gérant du bar d'agir manuellement sur les données utilisées par l'application. Autrement dit, elle garantira un accès facile à tout ce dont un patron a besoin pour faire tourner correctement son bar.

Exemples de manipulation : 
\begin{itemize}
\item modification manuelle du stock (bouteille cassée, ...)
\item Ajout d'un serveur.
\item modification manuelle de l'horaire du personnel (imprévu, serveur malade, ...)
\item modification manuelle d'une boisson (prix, description, ...) et/ou mise en promotion de celle-ci.
\item gestion des comptes clients (suppression ou ajout de compte, accord d'une réduction sur la prochaine consommation, ...)
\end{itemize}
\end{document}